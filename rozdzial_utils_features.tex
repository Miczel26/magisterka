\documentclass[printmode]{mgr}
%opcje klasy dokumentu mgr.cls zostały opisane w dołączonej instrukcji

%poniżej deklaracje użycia pakietów, usunąć to co jest niepotrzebne
	\usepackage{polski}       %przydatne podczas składania dokumentów w
%j. polskim 
%\usepackage[polish]{babel} %alternatywnie do pakietu
%polski, wybrać jeden z nich
\usepackage[utf8]{inputenc} %kodowanie znakĂłw, zaleĹźne od systemu

\usepackage[T1]{fontenc} %poprawne składanie polskich czcionek
\usepackage{amsmath}
%tutaj zaczyna się właśniwa treść	 dokumentu
\begin{document}

\chapter{Utils features}
Pomiar czasu został przeprowadzony za pomocą biblioteki HAL. Został wybrany 32-bitowy timer pozwalający na pomiar z dokładnością $\frac{1}{48000000}$ sekundy i łącznym czasem pomiaru $2^{32} * \frac{1}{48000000} \approx$ 90 sekund

\texttt{
&void TimerInit( TIM\_HandleTypeDef *timerHandle)\\
\{\\
timerHandle->Instance = TIM2;\\
\_\_TIM2\_CLK\_ENABLE();\\
timerHandle->Init.Prescaler = 1;\\
timerHandle->Init.CounterMode = TIM\_COUNTERMODE\_UP;\\
    timerHandle->Init.Period = 0xFFFFFFFF;\\
    timerHandle->Init.ClockDivision = 1;\\
}

\end{document}