\documentclass[printmode]{mgr}
%opcje klasy dokumentu mgr.cls zostały opisane w dołączonej instrukcji

%poniżej deklaracje użycia pakietów, usunąć to co jest niepotrzebne
%\usepackage{polski}       %przydatne podczas składania dokumentów w
%j. polskim 
\usepackage[polish]{babel} %alternatywnie do pakietu
%polski, wybrać jeden z nich
\usepackage[utf8]{inputenc} %kodowanie znakĂłw, zaleĹźne od systemu
\usepackage[T1]{fontenc} %poprawne składanie polskich czcionek

%pakiety do grafiki
\usepackage{graphicx}
\usepackage{subfigure}
\usepackage{psfrag}

%pakiety dodające dużo dodatkowych poleceń matematycznych
\usepackage{amsmath}
\usepackage{amsfonts}

%pakiety wspomagające i poprawiające składanie tabel
\usepackage{supertabular}
\usepackage{array}
\usepackage{tabularx}
\usepackage{hhline}

%pakiet wypisujący na marginesie etykiety równań i rysunków
%zdefiniowanych przez \label{}, chcąc wygenerować finalną wersję
%dokumentu wystarczy usunąć poniższą linię
\usepackage{showlabels}

%definicje własnych poleceń
\newcommand{\R}{I\!\!R} %symbol liczb rzeczywistych, działa tylko w
                        %trybie matematycznym
\newtheorem{theorem}{Twierdzenie}[section] %nowe otoczenie do
                                           %składania twierdzeń

%dane do złożenia strony tytułowej
\title{Tytuł‚ pracy magisterskiej}
\engtitle{Master thesis title}
\author{Michał Stroka}
\supervisor{dr inż. Grzegorz Budzyń, I-6}
%\guardian{dr hab. inż. Imię Nazwisko Prof. PWr, I-6} %nie używać
%jeśli opiekun jest tą samą osobą co prowadzący pracę

%\date{2008} %standardowo u dołu strony tytułowej umieszczany jest
%bieżący rok, to polecenie pozwala wstawić dowolny rok

%poniżej jest lista kierunków i specjalności na wydziale elektroniki,
%naleĚźy wybrać właściwe lub dopisać jeśli nie ma odpowiednich
\field{Automatyka i Robotyka (AIR)}

\specialisation{Komputerowe sieci sterowania (ARK)}

%tutaj zaczyna się właśniwa treść	 dokumentu
\begin{document}
\bibliographystyle{plabbrv} %tylko gdy uĚźywamy BibTeXa, ustawia polski
                            %styl bibliografii

%\maketitle 

\tableofcontents %spis treści

%poniżej znajduje się przykładowa treść dalszej części dokumentu,
%zainteresowanych zachęcam do rozszyfrowania frazy "Lorem ipsum" :)
\chapter{Wstęp}
Lorem ipsum dolor sit amet, consectetuer adipiscing elit. Nam
neque. Morbi interdum. Sed nibh nunc, imperdiet non, malesuada
blandit, tempor vel, odio. Aenean in leo ut nulla tempor
blandit. Praesent bibendum. Praesent varius imperdiet dolor. Etiam
urna eros, lacinia at, scelerisque quis, fringilla vel, lectus. Ut sit
amet dui ac pede lobortis varius. Integer mattis, neque non ornare
consequat, libero tellus luctus elit, a tempor pede massa nec
orci. Aenean ante lacus, feugiat in, lacinia sit amet, lobortis at,
leo. Class aptent taciti sociosqu ad litora torquent per conubia
nostra, per inceptos hymenaeos. Morbi ac ipsum eget est rutrum
porttitor. Duis sapien. Sed vel quam. Nullam volutpat facilisis magna.

\chapter{Cel pracy}

Celem pracy jest zbadanie możliwości biblioteki kryptograficznej wolfSSL na platformie STM32F2.
Porównane będzie gotowa implementacja z autorską próbą optymalizacji wybranych algorytmów szeregowych.

\chapter{Konfiguracja środowiska}

\chapter{AES}
Quisque nibh sapien, lobortis posuere, dignissim id, ultricies ut,
est. Cras nec nunc nec ligula placerat ullamcorper. Maecenas consequat
consequat nisi. Etiam varius. Vestibulum ante ipsum primis in faucibus
orci luctus et ultrices posuere cubilia Curae; Ut risus velit,
vulputate aliquet, tristique nec, ullamcorper et, justo. In hac
habitasse platea dictumst. Suspendisse rhoncus venenatis eros. Proin
nec lorem nec sem auctor lobortis. Quisque ac dolor. Suspendisse
potenti. Maecenas fermentum sem non neque. Cras nonummy commodo
libero. Sed in erat.


\section{Software}
Nulla quis enim ut erat rutrum feugiat. Suspendisse lacinia tempor
mi. Vestibulum nec lacus sed est rutrum cursus. Cras ultrices est eget
pede. Sed ullamcorper ultrices tellus. Nulla lectus. Nunc
consectetuer, quam quis sagittis vulputate, dui leo molestie augue, a
commodo tellus tortor a turpis. Nam et ipsum. Ut placerat aliquet
enim. Suspendisse potenti. Etiam volutpat tortor in mauris. Praesent
dapibus congue arcu.

\section{Hardware}
Aliquam quis sem. Phasellus tincidunt fringilla metus. Sed id
purus. Ut consectetuer urna nec sapien. Sed sed odio sed lectus
imperdiet ultricies. Suspendisse semper congue pede. Etiam facilisis
sapien nec dolor. Pellentesque sodales rutrum lorem. Aliquam tristique
diam ac lacus. Vestibulum ante ipsum primis in faucibus orci luctus et
ultrices posuere cubilia Curae; Duis sit amet pede. Suspendisse ut
dui. Aliquam massa quam, facilisis in, vulputate non, imperdiet quis,
diam. Nullam feugiat. Morbi non dolor et eros fringilla
gravida. Pellentesque rhoncus, odio ac fringilla varius, ante nunc
hendrerit massa, sed varius est est eu massa.


\begin{itemize}
\item Mauris nonummy lorem at orci.
\item Donec accumsan aliquam libero.
\item Donec fringilla ultricies diam.
\item Nulla venenatis est non ligula.
\item Morbi in mi convallis dolor accumsan egestas.
\item Sed euismod nibh in nulla.
\item Sed rhoncus lorem at lectus.
\item Pellentesque fermentum rutrum dui.
\end{itemize}

Proin euismod. Curabitur adipiscing ipsum ac augue. Maecenas hendrerit
tortor non velit suscipit laoreet. Aenean tempus. Nunc
convallis. Aenean sed erat. Etiam massa nulla, interdum pretium,
faucibus nec, sollicitudin vitae, lorem. Quisque vulputate cursus
pede. Phasellus enim ipsum, lacinia vel, sodales ac, faucibus et,
tortor. Aliquam hendrerit. Aliquam erat volutpat. Quisque dui lorem,
placerat eu, commodo et, condimentum sed, nibh. Nulla eu orci in nibh
pretium tincidunt. Pellentesque mi massa, fringilla eu, vehicula
consectetuer, sodales a, nisi.


\chapter{DES}
Nulla a nisl at nisl dignissim facilisis. In sit amet mi. Nunc
ullamcorper, ligula vel aliquam pretium, nulla mauris euismod ipsum,
sit amet tempus massa nulla non elit. Donec congue. Nulla
purus. Aenean dui orci, ornare quis, interdum sed, molestie ac,
tellus. Lorem ipsum dolor sit amet, consectetuer adipiscing
elit. Quisque mollis diam nec elit. Maecenas nec enim. Maecenas
facilisis urna quis arcu. Fusce posuere.


\appendix
\chapter{SHA}
Donec cursus nulla vitae pede. Etiam quam pede, aliquet ut,
pellentesque sed, sagittis non, est. Quisque egestas malesuada
risus. Maecenas ultricies libero a quam. Nullam feugiat arcu. Class
aptent taciti sociosqu ad litora torquent per conubia nostra, per
inceptos hymenaeos. In interdum, risus ut gravida sollicitudin, leo
sapien commodo dui, non consectetuer nisl nunc ac massa. Mauris a orci
in eros venenatis euismod. Curabitur orci. Quisque pharetra, dui sed
dignissim hendrerit, nibh ante malesuada eros, sed tincidunt magna
lorem a tellus. Aliquam erat volutpat. Aenean pulvinar, metus et
mattis dictum, massa lacus semper purus, quis vehicula augue mi et
leo. Ut eu ipsum. Sed dictum dapibus nisi. Cras mattis. Nulla sed
augue ac sem tempus condimentum.

\chapter{SHA-256}
Donec cursus nulla vitae pede. Etiam quam pede, aliquet ut,
pellentesque sed, sagittis non, est. Quisque egestas malesuada
risus. Maecenas ultricies libero a quam. Nullam feugiat arcu. Class
aptent taciti sociosqu ad litora torquent per conubia nostra, per
inceptos hymenaeos. In interdum, risus ut gravida sollicitudin, leo
sapien commodo dui, non consectetuer nisl nunc ac massa. Mauris a orci
in eros venenatis euismod. Curabitur orci. Quisque pharetra, dui sed
dignissim hendrerit, nibh ante malesuada eros, sed tincidunt magna
lorem a tellus. Aliquam erat volutpat. Aenean pulvinar, metus et
mattis dictum, massa lacus semper purus, quis vehicula augue mi et
leo. Ut eu ipsum. Sed dictum dapibus nisi. Cras mattis. Nulla sed
augue ac sem tempus condimentum.


\addcontentsline{toc}{chapter}{Bibliografia} %utworzenie w
                                             %spisietreści pozycji
                                             %Bibliografia

\bibliography{bibliografia} % wstawia bibliografię korzystając z pliku
                            % bibliografia.bib - dotyczy BibTeXa,
                            % jeĚźeli nie korzystamy z BibTeXa naleĚźy
                            % użyć otoczenia thebibliography

%opcjonalnie może się tu pojawić spis rysunków i tabel
% \listoffigures
% \listoftables
\end{document}

